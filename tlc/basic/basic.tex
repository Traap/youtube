% Shoutouts
% http://math.gillesgnacadja.info/:
% https://github.com/lervag

% https://github.com/lervag/vimtex
% https://github.com/lervag/vimtex/blob/master/doc/vimtex.txt

% https://github.com/lervag/wiki.vim
% https://github.com/lervag/wiki.vim/blob/master/doc/wiki.txt
%
% https://github.com/Traap/tlc-article/blob/master/tlc-article.cls
% YouTube: Step 1: Use document class article to create a basic LaTeX document.
% YouTube: Step 2: Rename article to tlc-tlc-article and uncomment clode.
% YouTube: Step 3: Add data/version.csv

\documentclass[12pt]{tlc-article}

\begin{document}
% YouTube: Step 4: Use tlcTitlePageAndTableOfContents
\tlcTitlePageAndTableOfContents{Basic \LaTeX\  Document}{Traap}{
  tlc-article exposes a command named tlcTitlePageAndTableOfContents to
  orchestrate  document layout. This \LaTeX\  document demonstrates layout
  design tlc-article uses in concert with facncyhdr to layout the first page
  with a document title, document author, document date, and document abstract
  with the headers and footers being empty.  Page two begins a consistent design
  the document header and document footer as well as the table of contents.
  Page three begins document content.
}

% YouTube: Step 4: Comment this line out.
% A \LaTeX\ document based on document class article.

% Youtube: Step 5: Uncomment this block.

\section{Intruduction}\label{sec:introduction}
A \LaTeX\  \nameref{sec:introduction} section using document class tlc-article.

\subsection{purpose}\label{sec:purpose}
A \LaTeX\  \nameref{sec:purpose} section using document class tlc-article.

\subsubsection{audience}\label{sec:audience}
A \LaTeX\  \nameref{sec:audience} section using document class tlc-article.

\section{Body}\label{sec:body}

A \LaTeX\  \nameref{sec:body} section using document class tlc-article.
\section{Conclusion}\label{sec:conclusion}
A \LaTeX\  \nameref{sec:conclusion} section using document class tlc-article.

% YouTube: Step 5: Enable debug

% tlcDebug must be the last line in your document to ensure all definitions and
% macros are fully expanded.
\tlcDebug

\end{document}
