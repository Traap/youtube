% {{{ simpledsv-l3 notes

% \csvreader[options]{file name}{assignments}{command list}
% \csvreader
%
%   options
%   [tabular = |r|l|l|
%   ,table head = \hline
%   ,table foot = \hline
%   ]
%
%  file name
%  {grade.csv}
%
%  assignments
%  {name=\name
%  ,givenname=\first-name
%  ,grade=\grade
%  }
%
%  command list
%  { \grade
%   &\firstname\~\name
%   &\csvcoliii
%  }
%
% -------------------------------------------------------------------------- }}}
% {{{ This file defines atomic elements.

\def\aeFile{data/elements.csv}

% -------------------------------------------------------------------------- }}}
% {{{ Use the ordinal position to define the column name.

\csvnames{aeNames}
{1=\aeAn % Atomic Number
,2=\aeSy % Symbol
,3=\aeNm % Name
,4=\aeAm % Atomic Mass
,5=\aeHc % CPK Hex Color
,6=\aeEg % Electrone gativity
,7=\aeAr % Atomic Radius
,8=\aeIe % Ionization Energy
,9=\aeEa % Electron Affinity
,10=\aeSs % Standard State
,11=\aeMp % Melting Point
,12=\aeBp % Boiling Point
,13=\aeDe % Density
,14=\aeGb % Group Block
,15=\aeYd % Year Discovered
}

% -------------------------------------------------------------------------- }}}
% {{{ Define a table style that is used to list the Atomic Hex Color.

\csvstyle{aeHexColorStyle}
{
  longtable=|C{1.5cm}|C{1.8cm}|L{3.0cm}|L{3.0cm}|L{2.5cm}|
  ,respect all
  ,column count=5
  ,table head=\hline
      Atomic Number
    & Symbol
    & Name
    & Std State
    & Hex Color
    \\\hline\hline\endhead
  ,late after line=\\\hline
  ,aeNames
}

% -------------------------------------------------------------------------- }}}
% {{{ Define a table style that is used to list the Year Discovered.

\csvstyle{aeYearDiscoveredStyle}
{
  longtable=|C{1.5cm}|C{1.5cm}|L{3.0cm}|L{1.5cm}|L{1.6cm}|
  ,respect all
  ,column count=5
  ,table head=\hline
      Atomic Number
    & Symbol
    & Name
    & Std State
    & Year Discovered
    \\\hline\hline\endhead
  ,late after line=\\\hline
  ,aeNames
}

% -------------------------------------------------------------------------- }}}
% {{{ Define a table style that is used to list boiling points, etc.

\csvstyle{aeAtomicDataStyle}
{
  longtable=|C{1.5cm}|C{1.5cm}|L{3.0cm}|L{1.5cm}|L{1.6cm}|L{1.6cm}|L{1.5cm}|
  ,respect all
  ,column count=5
  ,table head=\hline
      Atomic Number
    & Symbol
    & Name
    & Atomic Radius
    & Melting Point
    & Boiling Point
    & Density
    \\\hline\hline\endhead
  ,late after line=\\\hline
  ,aeNames
}

% -------------------------------------------------------------------------- }}}
% {{{ List the Year an Atomic Element was discovered.

\newcommand{\aeYearDiscovered}{%
  \IfFileExists{\aeFile}{%
    \csvreader
      [aeYearDiscoveredStyle%
      ,respect all%
      ,separator=pipe%
      ]%
    {\aeFile}
    {}
    {
       \aeAn
      &\aeSy
      &\aeNm
      &\aeSs
      &\aeYd
    }%
  }%
  {% Else
     aeFile [\aeFile] does not exist.
  }%
}%

% -------------------------------------------------------------------------- }}}
% {{{ List the CPK Hex Color for an Atomic Element.
%     Parameter 1 = [Gas | Liquid | State]

\newcommand{\aeListHexColors}[1]{%
  \IfFileExists{\aeFile}{%
    \csvreader
      [aeHexColorStyle%
      ,respect all%
      ,separator=pipe%
      ,filter equal={\aeSs}{#1}%
      ]%
    {\aeFile}
    {}
    {
       \aeAn
      &\aeSy
      &\aeNm
      &\aeSs
      &\aeHc
    }%
  }%
  {% Else
     aeFile [\aeFile] does not exist.
  }%
}%

% -------------------------------------------------------------------------- }}}
% {{{ Atomic Element general list.
%     argument 1: user provided column names to list

\newcommand{\aeGeneralList}[1]{%
  \IfFileExists{\aeFile}{%
    \csvreader
      [aeYearDiscoveredStyle%
      ,respect all%
      ,separator=pipe%
      ]%
    {\aeFile}
    {}
    {#1}%
  }%
  {% Else
     aeFile [\aeFile] does not exist.
  }%
}%

% -------------------------------------------------------------------------- }}}
% {{{ Atomic Element list.
%     arguments
%     1: csvstyle
%     2: column name
%     3: column value
%     4: columnscsvcol

\newcommand{\aeList}[4]{%
  \IfFileExists{\aeFile}{%
    \csvreader
      [#1%
      ,filter equal={#2}{#3}%
      ,respect all%
      ,separator=pipe%
      ]%
    {\aeFile}
    {}
    {#4}%
  }%
  {% Else
     aeFile [\aeFile] does not exist.
  }%
}%

% -------------------------------------------------------------------------- }}}
% {{{ These are helper definitions that are used to extract an individual column
%     attribute from the atomic elements file

\newcommand{\getAn}[1]{ \aePart{#1} {\aeAn} }
\newcommand{\getSy}[1]{ \aePart{#1} {\aeSy} }
\newcommand{\getNm}[1]{ \aePart{#1} {\aeNm} }
\newcommand{\getAm}[1]{ \aePart{#1} {\aeAm} }
\newcommand{\getHc}[1]{ \aePart{#1} {\aeHc} }
\newcommand{\getEg}[1]{ \aePart{#1} {\aeEg} }
\newcommand{\getAr}[1]{ \aePart{#1} {\aeAr} }
\newcommand{\getIe}[1]{ \aePart{#1} {\aeIe} }
\newcommand{\getEa}[1]{ \aePart{#1} {\aeEa} }
\newcommand{\getSs}[1]{ \aePart{#1} {\aeSs} }
\newcommand{\getMp}[1]{ \aePart{#1} {\aeMp} }
\newcommand{\getBp}[1]{ \aePart{#1} {\aeBp} }
\newcommand{\getDe}[1]{ \aePart{#1} {\aeDe} }
\newcommand{\getGb}[1]{ \aePart{#1} {\aeGb} }
\newcommand{\getYd}[1]{ \aePart{#1} {\aeYd} }

% -------------------------------------------------------------------------- }}}
% {{{ Extract one column from one rown from a csv file.
%     Arguments
%       #1 the atomic number to extract
%       #2 is the column name of the value to extract.

\newcommand{\aePart}[2]{%
  \IfFileExists{\aeFile}{%
    \csvreader
      [aeNames
      ,respect all%
      ,filter equal={\aeAn}{#1}%
      ,separator=pipe%
      ]%
    {\aeFile}%
    {}%
    {#2}%
  }%
  {% Else
    aeFile [\aeFile] does not exist.
  }%
}%

% -------------------------------------------------------------------------- }}}
