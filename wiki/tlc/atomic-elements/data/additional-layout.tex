% {{{ This file defines atomic elements.

\def\aeFile{data/elements.csv}

% -------------------------------------------------------------------------- }}}
% {{{ Define column na {\aeFile}mes used by csvreader for atomic elements.

\csvnames{aeNames}{
    1=\aeAn
   ,2=\aeSy
   ,3=\aeNm
   ,4=\aeAm
   ,5=\aeHc
   ,6=\aeEg
   ,7=\aeAr
   ,8=\aeIe
   ,9=\aeEa
  ,10=\aeSs
  ,11=\aeMp
  ,12=\aeBp
  ,13=\aeDe
  ,14=\aeGb
  ,15=\aeYD
}

% -------------------------------------------------------------------------- }}}
% {{{ Define a table style that is used to list all atomic elements .

\csvstyle{aeStyle}{
  longtable=|C{1.5cm}|C{1.8cm}|L{3.0cm}|L{3.0cm}|L{2.5cm}|
  ,respect all
  ,column count=5
  ,table head=\hline
      Atomic Number
    & Symbol
    & Name
    & SS
    & Hex Color
    \\\hline\hline\endhead
  ,late after line=\\\hline
  ,aeNames
}

% -------------------------------------------------------------------------- }}}
% {{{ Extract one row from a csv file.
%     Arguments
%       #1 the atomic number to extract
%       #2 is the column name of the value to extract.

\newcommand{\aePart}[2]{%
  \IfFileExists{\aeFile}{%
    \csvreader
      [respect all%
      ,column count=15%
      ,separator=pipe%
      ,filter equal={\aeAtomicNumber}{#1}%
      ]%
    {\aeFile}%
    {\global\edef\gAtomicNumber{\aeAtomicNumber}#2}%
    {%
      \aeAn
      &\aeSy
      &\aeNm
      &\aeAm
      &\aeHc
      &\aeEg
      &\aeAr
      &\aeIe
      &\aeEa
      &\aeSs
      &\aeMp
      &\aeBp
      &\aeDe
      &\aeGb
      &\aeYD
    }%
  }%
  {% Else
    aeFile [\aeFile] does not exist.
  }%
}%

% -------------------------------------------------------------------------- }}}
% {{{ This function lists all atomic elements

\newcommand{\aeList}{%
  \IfFileExists{\aeFile}{%
    \csvreader
      [aeStyle%
      ,respect all%
      ,separator=pipe%
      ]%
    {\aeFile}
    {}
    {
       \aeAn
      &\aeSy
      &\aeNm
      &\aeSs
      &\aeHc
    }%
  }%
  {% Else
     aeFile [\aeFile] does not exist.
  }%
}%

% -------------------------------------------------------------------------- }}}
% {{{ Define a table style that is used to list atmoc elements hex colors.

\csvstyle{aeHexColorStyle}{
  longtable=|C{1.5cm}|C{1.8cm}|L{3.0cm}|L{3.0cm}|L{2.5cm}|
  ,respect all
  ,column count=5
  ,table head=\hline
      Atomic Number
    & Symbol
    & Name
    & SS
    & Hex Color
    \\\hline\hline\endhead
  ,late after line=\\\hline
  ,aeNames
}

% -------------------------------------------------------------------------- }}}
% {{{ This function selects all records in the atomic element file that match
% {{{ These are helper definitions that are used to extract an individual column
%     attribute from the atomic elements file

\def\getAn{ \aePart {\aeKey} {\aeAn}}
\def\getSy{ \aePart {\aeKey} {\aeSy}}
\def\getNm{ \aePart {\aeKey} {\aeNm}}
\def\getAm{ \aePart {\aeKey} {\aeAm}}
\def\getHc{ \aePart {\aeKey} {\aeHc}}
\def\getEc{ \aePart {\aeKey} {\aeEc}}
\def\getEg{ \aePart {\aeKey} {\aeEg}}
\def\getAr{ \aePart {\aeKey} {\aeAr}}
\def\getIe{ \aePart {\aeKey} {\aeIe}}
\def\getEa{ \aePart {\aeKey} {\aeEa}}
\def\getOs{ \aePart {\aeKey} {\aeOs}}
\def\getSs{ \aePart {\aeKey} {\aeSs}}
\def\getMp{ \aePart {\aeKey} {\aeMp}}
\def\getBp{ \aePart {\aeKey} {\aeBp}}
\def\getDe{ \aePart {\aeKey} {\aeDe}}
\def\getGb{ \aePart {\aeKey} {\aeGb}}
\def\getYD{ \aePart {\aeKey} {\aeYD}}

% -------------------------------------------------------------------------- }}}
% aenSs
%     Parameter 1 - Active | Closed | New | Parley.

\newcommand{\aeListHexColors}[1]{%
  \IfFileExists{\aeFile}{%
    \csvreader
      [aeHexColorStyle%
      ,respect all%
      ,separator=pipe%
      ,filter equal={\aeSs}{#1}%
      ]%
    {\aeFile}
    {}
    {
       \aeAn
      &\aeSy
      &\aeNm
      &\aeSs
      &\aeHc
    }%
  }%
  {% Else
     aeFile [\aeFile] does not exist.
  }%
}%

% -------------------------------------------------------------------------- }}}
